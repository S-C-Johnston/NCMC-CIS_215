%&tex
\documentclass{article}

\usepackage{graphicx}[draft]
\usepackage{color}
\graphicspath{{./img/}}
\usepackage{geometry}
\usepackage{hyperref}
\usepackage{subcaption}
\usepackage{float}
\usepackage{listings}

\title{Exercise 04 -- Normalization}
\author{Stewart Johnston\\
  {CIS 215 -- Database Design}\\
  {NCMC}\\
  {\texttt{johnstons1@student.ncmich.edu}}
}
\date{\today}

\begin{document}

\maketitle

The invoice given us to normalize the data and put into a relational
database is not naturally in a normalized form. Several different tables
need to be split off in order to accurately model the data while also
making it simple to work on.

From \includegraphics{norm5.jpg}

The tenets of first normal form include:
\begin{enumerate}
	\item Atomic values for any given field of any given tuple. In
		more plain English, for any intersection of row and
		column, therein will be only one value of the
		attribute's type or domain.
		
		It is often phrased as having ``No Repeating Groups''.
		The important part of this is consideration is that if
		multiple pieces of data are introduced with the same
		type/domain, they must be uniquely identifiable in
		the row without asking for the nth instance of that
		attribute.  ``Address_1'' and ``Address_2'', for
		example, would not be appropriate.  However,
		``shipping_address'' and ``billing_address'' may be
		appropriate, depending on the business rules and the
		cardinality of the relationship.
		
		A 1:n cardinality, for instance, would not be
		appropriate to denormalize in this way. If any one
		entity had an exact number of named variants of the same
		domain (e.g. work and cell number), pulling them into
		the main relation may be appropriate. Your mileage may
		vary.

	\item Each tuple in the relation must be uniquely identifiable
		with a primary key. A primary key can be an arbitrary
		auto-incrementing value, which is quite often the case.
		A primary key can be composed of multiple pieces of a
		tuple's data, as long as this data will never collide on
		multiple tuples. It is recommended that this primary key
		be composed of values which are likely never to change
		but also do not have confidentiality/security
		requirements. (e.g., a birth date in combination with
		other data may be appropriate, but a social security
		number would not be).

	\item Groups of similar data should be wrapped in their own
		tables, \emph{especially} if they have any cardinality
		other than 1:1 with any other data belonging to the same
		entity.

		In the exercise, individual products have a many:many
		cardinality with invoices, so products are best stored
		in their own table. The quantity of any given product on
		any given invoice is best stored as an attribute of an
		associative table between the invoice primary key and
		the product primary key, since those are the
		determinants of the quantity.

\end{enumerate}

\includegraphics{Ex04-Normalization-Stewart_Johnston.gv.png}

\end{document}
