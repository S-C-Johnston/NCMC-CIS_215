%&tex
\documentclass{article}

\usepackage{graphicx}[draft]
\usepackage{color}
\graphicspath{{./img/}}
\usepackage{geometry}
\usepackage{hyperref}
\usepackage{subcaption}
\usepackage{float}
\usepackage{listings}

\title{Exercise 04 -- Normalization}
\author{Stewart Johnston\\
  {CIS 215 -- Database Design}\\
  {NCMC}\\
  {\texttt{johnstons1@student.ncmich.edu}}
}
\date{\today}

\begin{document}

\maketitle

The invoice given us to normalize the data and put into a relational
database is not naturally in a normalized form. Several different tables
need to be split off in order to accurately model the data while also
making it simple to work on.

From \includegraphics{norm5.jpg}

The tenets of first normal form include:
\begin{enumerate}
	\item Atomic values for any given field of any given tuple. In
		more plain English, for any intersection of row and
		column, the value should be immediately usable for
		business rules without further computation on that
		value.

		Example: One may have both a work and cell phone number,
		but in storing the data about this in a relational
		database, a column in first normal form will contain
		only one value of its type.
\end{enumerate}

\includegraphics{Ex04-Normalization-Stewart_Johnston.gv.png}

\end{document}
